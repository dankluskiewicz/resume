\documentclass[10pt]{article} %Sets the default text size to 11pt and class to article.
%------------------------Dimensions--------------------------------------------

\headheight=0pt %1in margins at top and bottom (1 inch is added to this value by default)
\headsep=0pt %Increase to increase white space in between headers and the top of the page
\textheight=9.0in %How tall the text body is allowed to be on each page
\usepackage{shading}
\usepackage[colorlinks=true, urlcolor={blue}]{hyperref}
\usepackage{tcolorbox}
\usepackage[top=.5in, bottom=.5in, left=.8in, right=.8in]{geometry}
\usepackage{enumitem}
\setlist{nosep}

\tcbset{
    frame code={}
    center title,
    left=0pt,
    right=0pt,
    top=0pt,
    bottom=0pt,
    colback=gray!40,
    colframe=white,
    width=\dimexpr\textwidth\relax,
    enlarge left by=0mm,
    boxsep=5pt,
    arc=0pt,outer arc=0pt,
    }

\begin{document}
\pagenumbering{gobble}
%These two pieces of code tell LaTeX that everything that goes in between these tags is what you want displayed as your actual document.

%%%%%%%%%%%%%%%%%%%%%%%%%%%%%%%%%%%%
\centerline{{\Huge \sc Dan Kluskiewicz}}  %Makes whatever text you put in parenthesis move to the center
%Prevents the following text from being indented
%This is the same as a return in Latex
\centerline{Index, WA \textbullet \hspace{5pt} (215) 421-9496}
\centerline{\href{mailto:dklus@uw.edu}{dklus@uw.edu} 
\textbullet \hspace{5pt}  \url{www.linkedin.com/in/dkluskiewicz}
\textbullet \hspace{5pt}  \url{www.dankluskiewicz.com}
}

\noindent
%\textbf{Objective:} I would like a field-work intensive job in geophysics.

\vspace{-2mm}
\noindent
\line(1,0){497}\\
\vspace{-3.5mm}

%%%%%%%%%%%%%%%%%%%%%%%%%%%%%%%%%%%%
\begin{tcolorbox}
\noindent \centerline{\large \bf University of Washington \hfill 2015} \\
\textbf{M.S. in Geophysics} \hfill GPA: 3.94 \\
{\normalsize \hfill NSF Graduate Research Fellowship}
\end{tcolorbox}
\vspace{-1.5mm}

\vspace{-1mm}
\begin{tcolorbox}
\noindent \centerline{\large \bf Pennsylvania State University, Schreyer Honors College \hfill 2010}
\textbf{B.S. in Physics and Mathematics}\hfill GPA: 3.99\\
{\normalsize \hfill Highest Distinction, Class rank \textbf{1} out of \textbf{519} (College of Science)}
\end{tcolorbox}
\vspace{-1.5mm}


\vspace{-2.5mm}
\noindent
\line(1,0){497}\\
\vspace{-3.5mm}

%%%%%%%%%%%%%%%%%%%%%%%%%%%%%%
%%%%%%%%%%%%%%%%%%%%%%%%%%%%%% Skills

\noindent

\noindent {\bf Languages:}
 Python (incl. Numpy, Matplotlib, Pandas, GeoPandas, and Scikit-Learn), \LaTeX, (some R and Matlab)

\noindent {\bf Skills:}
Research, Data Wrangling, Mathematical Modeling, Problem Solving, Probability \& Statistics, Machine Learning, Technical Writing, Data Visualization, Linear Algebra,
GIS, Numerical Methods for Differential Equations, Signal Processing, Teaching


\noindent {\bf Tools:}
Git, AWS Cloud Computing, MS Word, Excel, Powerpoint, Linux, OSX, Windows


\vspace{-2mm}
\noindent
\line(1,0){497}\\
\vspace{-3.5mm}

%%%%%%%%%%%%%%%%%%%%%%%%%%%%%%
%%%%%%%%%%%%%%%%%%%%%%%%%%%%%% Jobs
\noindent {\Large \bf Work Experience}

\begin{small}

\noindent
\textbf{Northwest Management, Inc} ---  Moscow, ID (working in Index, WA) \hfill \textbf{September, 2018 - Present}$\,$

\noindent
\,\,
\textbf{\textit{Geospatial Programmer}} 

\noindent
Led research and development for a statistical model that predicts individual-tree forest inventory from a combination of field samples and aerial imagery and Lidar. 

\begin{itemize}[leftmargin = .35in,     noitemsep]
\item Designed and programmed a modeling workflow that organizes terabytes of remote sensing data, performs efficient
model experiments in the vicinity of our training (field sample) data, evaluates the performance of an optimized model,
and then implements the optimized model across a project. 
\item Delivered forest inventories and documentation (including validation statistics) for multiple large projects while
continuing to refine all components of our modeling workflow. 
\item Collaborated with foresters and biometricians to design efficient field sampling protocols 
\item Presented our inventory products and validation data to clients, as well as at multiple conferences domestic and abroad.
\end{itemize}

\vspace{1mm}
\noindent
\textbf{Talus Analytics} --- Boulder, CO  \hfill \textbf{June,  2016 - April, 2018}

\noindent
\,\,
\textbf{\textit{Quantitative Researcher}}  \hfill October,  2017 - April, 2018$\,$

\noindent
\,\,
\textbf{\textit{Junior Quantitative Researcher}} \hfill June, 2016 - September, 2017$\,$

\begin{itemize}[leftmargin = .35in, 	noitemsep]
\item Developed predictive models for flooding and wildfire spread
\begin{itemize}[leftmargin = .3in, 	noitemsep]
\item Researched existing methods and theory
\item Proposed a novel geometric method for probabilistic flood-hazard assessment, and led a collaborative effort to implement it in Python.
\item Implemented a simplified version of an existing static fire model (coded from scratch), and extended it to include time-dependent predictions
\end{itemize}
\item Developed and implemented a wildfire accumulation model that delineates geographic areas based on wildfire hazards and predicts probable maximum losses within them
\item Wrote the methods section for a FEMA publication on rapid flood modeling
\item Wrote and extensively edited company reports on flood modeling and resilience investments
\end{itemize}

\vspace{-2mm}
\noindent
\line(1,0){497}\\
\vspace{-4mm}

%%%%%%%%%%%%%%%%%%%%%%%%%%%%%%
%%%%%%%%%%%%%%%%%%%%%%%%%%%%%% activities
\noindent {\Large \bf Publications}
\begin{itemize}
\item Kluskiewicz, Dan and others. (2017). Sonic methods for measuring crystal orientation fabric in ice, and results from the West Antarctic ice sheet (WAIS) Divide. Journal of Glaciology. 63. 1-15. 10.1017/jog.2017.20. 
\item Longenecker, Herbert and others. (2019). A Rapid Flood Risk Assessment Method for Response Operations and Non‐subject‐matter‐expert Community Planning. Journal of Flood Risk Management. 13. 10.1111/jfr3.12579. 
\end{itemize}

\end{small}
\end{document}